\documentclass{beamer} 

% Michael Maier, 2013.
% CC-BY-SA

\usepackage[utf8]{inputenc}
\usepackage[ngerman]{babel}

\title{OpenVPN} 
\author{Michael Maier \textless Michael.Maier@student.tugraz.at\textgreater} 
\date{20. April 2013} 

\usetheme{Antibes}

\hypersetup{colorlinks=true,urlcolor=blue,linkcolor=white}

%\usebackgroundtemplatei{
%\includegraphics[width=\paperwidth,
%height=0.8\paperheight]{mag_map.png}
%}

\begin{document}

%\maketitle

\begin{frame} 


\begin{figure}
  \centering
  \includegraphics[width=.1\textwidth]{openvpn.png}
\end{figure}

\begin{center}
\Huge{OpenVPN\\}
\end{center}

\begin{center}
\Large{\emph{Sicher zwischen Netzen}}
\end{center}

\end{frame}



\begin{frame}{Vorstellung}

  \begin{itemize}
    \item Michael Maier \textless \href{mailto:Michael.Maier@student.tugraz.at}{Michael.Maier@student.tugraz.at}\textgreater
    \item Telematik-Student an der TU Graz 
    \item Linux, Open-Source begeistert seit 2004
    \item ZID der Montanuniversität 2006-2007
    \item Nebenbei freiberuflich tätig seit 2009
	    \pause
    \item Leite den Grazer OSM-Stammtisch seit May 2011
%    \begin{itemize}
%      \item OSM-username: \emph{\href{http://www.openstreetmap.org/user/species}{species}}
%      \item Mapping-area Graz, Leoben on bicycle, motorbike or on public transport
%    \end{itemize}
  \end{itemize}
\end{frame}

\section{VPN - Virtual Private Network?}

\begin{frame}{Struktur von Netzwerken}

	
	Bild - Internet (Cloud) - Firewall - Internes Netz (Intranet)

\end{frame}




\begin{frame}{Privates Netz - Öffentliches Netz}

Intern biete ich mehr Dienste als von Aussen an

\begin{itemize}
  \item Fileserver
  \item VoIP-Server
  \item Wartungszugänge zu Servern/PCs
  \item Internen DNS
  \item Bsp Uni: Zugang zu Bibliotheken auf IP-Adressbasis
\end{itemize}
\end{frame}

\begin{frame}{Externer Laptop?}
	Soll ins interne Netz - Warum?

	Will außerhalb dieselben Services nutzen können, wie wenn ich daheim wäre
\begin{itemize}
  \item Zugriff auf Fileserver
  \item Zugriff auf Intranet-Seiten
  \item Surfen hinter gesicherter Firewall
\end{itemize}
\pause
Lösung?
  \begin{itemize}
    \item Virtual Private Netzwork!
  \end{itemize}
\end{frame}

\begin{frame}{VPN-Tunnel}

	,,Durchtunnelt'' die Firewall

	\pause

	Wie sieht das in der Praxis aus

\begin{itemize}
	\item Bekomme weitere (virtuelle) Netzwerkkarte
	\item Diese tut so, als wäre ich in meinem Privaten Netz
		\pause
	\item Routing über die virtelle Netzwerkkarte:
  \begin{itemize}
	  \item Entweder nur der Traffic fürs Intranet (192.168.0.*)
	  \item Oder alles, ich surfe dann über die Firma ins Internet
  \end{itemize}
\end{itemize}
  \pause
    Sehr praktisch, wenn in meinem WLAN/Hotel/... nicht alle Dienste erlaubt sind!
\end{frame}

\begin{frame}{Weitere Anwendung: Broadcast-Dienste}
	Broadcast ... 255.255.255.255 

	Definiert als DIESES Netzwerk - um solche Dienste zu nutzen muß ich in diesem Netzwerk sein!

	\pause

	Schreit förmlich nach VPN!
\end{frame}

\section{OpenVPN}

\begin{frame}{OpenVPN}
\begin{itemize}
  \item GPL-Lizensierte Software
  \item Eigenes Protokoll, OpenSSL-basiert 
  \begin{itemize}
    \item Linux, Android, *WRT
    \item *BSD
    \item MacOS/iOS
    \item Solaris/QNX
    \item Windows 2K/XP/Vista/7
  \end{itemize}
  \item Einfache Installation
  \item ,,Einfache'' Konfiguration ... \pause Simple Config-Files
  \item Automatisches Failover möglich
\end{itemize}

\end{frame}

\begin{frame}{OpenVPN Modi}
	Authentifizierung
  \begin{itemize}
    \item PSK (Pre-Shared Key)
    \item Benutzername/Passwort
    \item Zertifikatsbasiert
  \end{itemize}
\pause
	Netzwerkmodi:
\begin{itemize}
	\item layer-2 based Ethernet (TAP)
	\item layer-3 based IP tunnel (TUN)
\end{itemize}

\end{frame}

\begin{frame}{Aus der Praxis: Beispiel zur Nutzung von OpenVPN}

	Proprietäre Geräte, die nur über Broadcast kommunizieren (Waren für reinen LAN-Betrieb ausgelegt) sollen ins WLAN.
	\pause
	Ja und?
	\pause
	Broadcast \& WLAN: Böse!

  \begin{itemize}
    \item Jedes Paket muß zu jedem Teilnehmer gesendet werden, auch wenn derjenige es garnicht braucht.
    \item 34 Clients, die miteinander reden wollen ... 
    \item 2 Broadcast-Pakete / Sekunde je 500\,B.
	    \pause
    \item Ergibt: $68$ Pakete/s auf $33$ Zielrechner : $2244$ Pakete/sec Worstcase $\Rightarrow$ No-Go!
	    \pause
    \item JEDES Gerät im WLAN wird mit Grundlast von 34\,KB/s bombardiert! $\Rightarrow$ bei 20 Geräten am AP~680\,KB/s
  \end{itemize}


\end{frame}

\begin{frame}{Lösung: OpenVPN!}
	Da wir nicht alle anderen Clients im WLAN (Laptops, Handys, Tablets,...) mit unseren Pakete belästigen wollen, spannen wir innerhalb des WLANs ein VPN auf:
	\pause
  \begin{itemize}
    \item Ein zentraler OpenVPN-Server
    \item Broadcast-Pakete in den Tunneln werden nur an unsere Clients gesendet
    \item Einrichten von Subnetzen mit Firewall-trennung
    \item Filtern unwichtiger Broadcasts
    \item Komprimierung der Daten durch OpenVPN
  \end{itemize}
\end{frame}

\begin{frame}{Realisierung:}
  \begin{itemize}
	  \item Gumstix (r) an den LAN-Clients, die sich ins WLAN einwählen
	  \item OpenVPN-Server in einer VM, ins WLAN-Netz geroutet
	  \item Clients in 3 Subnetze getrennt
	  \begin{itemize}
		  \item 3 OpenVPN-Endpunkte am Server
		  \item Interfaces mit brctl gebridged
		  \item Filternde Firewall zwischen den Subnetzen
		  \item Ein Teil der Broadcast-Paket war unnötig, 6/7 gedroppt.
	  \end{itemize}
  \end{itemize}
\end{frame}

\begin{frame}{Hilfe}

Links

\begin{itemize}
	\item \url{http://openvpn.net}
	\item Manpage (!): \emph{man 8 openvpn}
\end{itemize}

\end{frame}

\section{Ende}

\begin{frame}{Vielen Dank für die Aufmerksamkeit!}

  Folien zu OpenVPN auf den Grazer Linuxtagen, 20.4.2013.
\vspace{1cm}

Folien unter: \includegraphics[width=1cm]{cc-by-sa.png}.
\vspace{1cm}

Erstellt mittels \LaTeX Beamer, Quelltext: \href{https://github.com/species/vortrag-openvpn-glt13}{Github}.
\vspace{1cm}

\href{mailto:michael.maier@student.tugraz.at}{Michael Maier}

Twitter: \href{https://twitter.com/osmgraz}{@osmgraz}
\end{frame}

\end{document}
